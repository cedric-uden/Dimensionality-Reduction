
To round up the theoretical premises required for this topic, we will become aware of the boundaries of intuitive mathematical concepts which result in highly counter-intuitive behaviours in high dimensional space, as well as their practical solution approaches.

\subsubsection{Euclidean distance \& sparse matrices}

An important aspect which is frequently utilised in various machine learning methods is to evaluate the euclidean distance between two points in a high-dimensional space.
While the concept is simple to understand and illustrate in two or three dimensions, its behaviour in a high-dimensional space changes dramatically and it becomes heavily counter-intuitive to get a hold off.

To discover its behavioural implications, we are considering an n-dimensional space and observing the impact of increasing dimensions between seemingly contrastable points. 
For this .... \todo{finish this section \& consider the sparse matrices section}



\vspace{4mm}

\noindent
\begin{minipage}[c]{0.45\linewidth}

\subsubsection{Geodesic distance}

\begin{center}
	\textit{Why geodesic distances are important and how they are read}
\end{center}



\end{minipage}\hfill%
\begin{minipage}[c]{0.45\linewidth}

\subsubsection{Eigenvalues and eigenvectors}

\begin{center}
	\textit{Quickly recapitulate eigenvectors and explain why they are relevant}
\end{center}


\end{minipage}%

