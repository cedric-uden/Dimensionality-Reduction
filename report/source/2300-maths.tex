
To round up the theoretical premises required for this topic, we will become aware of the boundaries of intuitive mathematical concepts which result in highly counter-intuitive behaviours in high dimensional space, as well as their practical solution approaches.

\subsubsection{Euclidean distance \& sparse matrices}

An important aspect which is frequently utilised in various machine learning methods is to evaluate the euclidean distance between two points in a high-dimensional space.
While the concept is simple to understand and illustrate in two or three dimensions, its behaviour in a high-dimensional space changes dramatically and it becomes heavily counter-intuitive to get a hold off.

Most notably, if you pick two random points in a unit hypercube, the higher the dimensions, the higher the average distance between these two points will be \cite{HandsOnMLCh8}.
This implies that, the higher the dimensionality of a dataset, the higher the chance of overfitting.
In this scenario, the matrices which represent a high-dimensional dataset are called sparse matrices.

\vspace{2mm}

\noindent
\begin{minipage}[t]{0.45\linewidth}

\subsubsection{Geodesic distance}

The geodesic distance between two points can be thought of as the minimum number of nodes between two nodes in a graph \cite{HandsOnMLCh8}.

\vspace{1mm}

\begin{center}
	\renewcommand{\tikzscale}{0.45}
	\begin{tikzpicture}[scale = \tikzscale]
	\draw[iwiPurple, dashed, line width = 0.50mm]   plot[smooth,domain=-8:8] (\x, {-0.1*(\x)^2});

	\filldraw[color=hkaRed] (-7,-7) circle (5pt);
	\filldraw[color=hkaRed] (-5.5,-5) circle (5pt);
	\filldraw[color=hkaRed] (-3.5,-3) circle (5pt);
	\filldraw[color=hkaRed] (0,-1) circle (5pt);
	\filldraw[color=hkaRed] (3.5,-3) circle (5pt);
	\filldraw[color=hkaRed] (5.5,-5) circle (5pt);
	\filldraw[color=hkaRed] (7,-7) circle (5pt);

    \path[draw, dotted,thick]    (-6,-7) -- (6,-7);

    \node at (0,1.25) {\textcolor{iwiPurple}{Geodesic Distance}};
    \node at (0,-6) {Euclidean Distance};


\end{tikzpicture}

	\vspace*{2mm}
	\captionsetup{justification=centering,type=htypei}
	\captionof{figure}{Geodesic distance\\in comparison to Euclidean distance\cite{DRUnsupervisedNearestNeighbors}}
	\label{fig:geodesicdistance}
\end{center}

While the Euclidean distance can be thought of as the beeline between two cities, the geodesic distance would be the driving route.

\end{minipage}\hfill%
\begin{minipage}[t]{0.45\linewidth}

\subsubsection{Eigenvalues and eigenvectors}

\begin{center}
	\textit{Quickly recapitulate eigenvectors and explain why they are relevant}
\end{center}


\end{minipage}%

