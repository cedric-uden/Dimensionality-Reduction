\Gls{pca} has induced a fluctuating history during the span of the past century.
First appearances date back as early as the beginning of the $19^{\text{th}}$ century. 
During this time, the method has undergone several iterations while being reinvented multiple times across different disciplines \cite{jolliffe2016principal}.\bigskip


It is generally accepted that the first two techniques, which both describe two considerably different approaches to \gls{pca}, were developed by \citeauthor{pearson1901liii} and \citeauthor{hotelling1933analysis}. 
The reports evolved in the years of \citeyear{pearson1901liii} and \citeyear{hotelling1933analysis} respectively \cite{Jolliffe2002book}.

Karl Pearson, an English mathematician, approached the problem geometrically \cite{pearson1901liii}.
He attempted to identify the planes that best fit a set of points in a p-dimensional space.
This concept closely follows the fundamentals of linear regression problems.

Harold Hotelling on the other hand, an American mathematician, approached the problem using algebraic derivation \cite{hotelling1933analysis}.
His method consists of using Lagrange multipliers to find the minima and maxima of the problem.
The resulting matrices representing the data are then expressed as an eigenvalue / eigenvector problem.\medskip


Both approaches however, had an essential common ground: the computations beyond 4 dimensions are astoundingly complex.
Even if \citeauthor{pearson1901liii} was convinced that his calculations were feasible albeit cumbersome, his prediction turned out to be wrong.
The rise of \gls{pca} as a crucial statistical tool has not begun until much later, when computers become more and more ubiquitous in our world.\bigskip


Further worth mentioning, Rao \cite{rao1964use} contributed important new ideas concerning the use, interpretations, and extensions of \gls{pca}.
This was not until much later in \citeyear{rao1964use}.
Around the same time in \citeyear{jeffers1967two}, Jeffers \cite{jeffers1967two} showcased the practicality of \gls{pca} in two examples. 
These examples demonstrated that there the possibilities in \gls{pca} are not restricted to simple \acrlong{dr} \cite{Jolliffe2002book}.

% \item Good overview by \cite{jolliffe2016principal}
% \item First approaches made in 1901 using simple projections. \cite{pearson1901liii} (united kingdom)
% Jolliffe book section 3.2 - linear regression approach (p.34)
% \item Independently came \cite{hotelling1933analysis} in 1933 (east coast, Princeton) 
% Similar approach to Lagrange multipliers and ending in an eigenvalue / eigenvector problem. (Jolliffe book section 1.2 - p.7)
% \item Computationally feasible until much later
% \item Has been used and reinvented multiple times in many different disciplines \cite{jolliffe2016principal}


% \begin{itemize}
% 	\item First introduced by \cite{scholkopf1998nonlinear}
% 	\item According to this ...%\cite{}
% \end{itemize}

% $$\bigo{TBA}$$

% \clearpage

% \begin{center}
% 	\textit{Not sure what comes here, but most certainly going to be two pages}
% \end{center}
