\Gls{mva} has not widely been exploited until the rise of computers in modern times \cite{Jolliffe2002book}.
This becomes apparent when contemplating its intensive calculations which reach further than our human imagination deems possible.
As we will study in this paper, the theory of \gls{dr} in \gls{mva} is considerably older than its applications.
Thanks to modern technologies and computational resources, \gls{dr} is now well entrenched in most statistical applications.
Accompanying its progress, substantial revelations have emerged in recent years which we will elaborate in greater detail in this work.
\bigskip


The primary goal of \gls{dr} is to maximise the variance between \glspl{latentVariable} of a given data set.
This allows us to transform a large dimensional data set into its \gls{intrinsicDimension}.
Our goal along this journey is to minimise information loss while increasing interpretability \cite{jolliffe2016principal}.
This is made possible by the condition that naturally occurring systems often exhibit dominant patterns.
\Gls{dr} also has further secondary effects which comes with advantages that will be established alongside this work.
\medskip

To achieve this, we will begin by stating the circumstances surrounding the domain of \gls{dr}.
Based on this insight, we will then discuss the origins and some solution approaches in \gls{pca}, one of the most popular techniques in \gls{dr}.
We will look at different algorithms, their benefits and implementation.
The implementation is based off the popular machine learning library \gls{scikit}.
Afterwards, we will look at specific applications how it can be used and then examine common pitfalls to avoid.
To close off, we will briefly look at an entirely different technique to be able to better narrow down which technique is beneficial in each scenario.
\bigskip


Finally, it is worth noting that this work merely portrays a brief outline of the subject and cannot precisely cover the vast research conducted during the past century.
Additionally, we will simply consider the domain of the real numbers $\mathbb{R}$ during this elaboration.


% \item Like many other multivariate methods, it was not widely used until the rise of computers. 
% It is now well entrenched in most statistical applications. \cite{Jolliffe2002book}
% \item We will explore how to transform a large dimensional datasets into its \gls{intrinsicDimension}
% \item We will take a look at different algorithms, their benefits and fallacies as well as their implementation
% \item Increase intepretability while minimizing information loss \cite{jolliffe2016principal}
% \item Do not forget to mention that we are only considering real numbers and this is also possible in the complex number space with certain more caveats
% \item Naturally occurring systems often exhibit dominant patterns
% \item Can be used to de-noise data sets (\cite{brunton2019data} - 1.1)
