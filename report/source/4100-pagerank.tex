One famous application of \acrlong{pca} \cite{deisenroth2020mathematics} is the PageRank algorithm proposed by \mycite{page1999pagerank}.
This publication led to the beginning of the Google search engine which demonstrates its efficacy and success.

The primary motivation to build a new search engine arose as the authors saw the lack of success in human maintained lists which failed to keep up with the demands from the rise of the world wide web.
Other attempts such as automated search engines that rely on keywords generally returned unsatisfactory matches.
To make matters worse, both attempts were easy to manipulate by advertisers.
\bigskip


The algorithm, already back in the 1990's, had to deal with hundreds of millions of entries in its data set \cite{brin1998anatomy}.
Simultaneously, it had to answer tens of millions of queries a day.
To achieve such feature, \citeauthor{page1999pagerank} calculate the PageRank using a simple iterative algorithm which corresponds to the principal eigenvectors of the normalised link matrix.
The feasibility of these computations was stated to be possible on medium sized workstations \cite{page1999pagerank}.

To maintain the accurate results, the paper emphasises the importance of crawling and indexing the data as a crucial application of the search engine.
To maintain numerical stability in the search results, the crawled data requires to be precisely parsed across the myriad of HTML structures.
