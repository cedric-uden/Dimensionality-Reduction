Being acquainted with the premise, we now possess a coarse understanding of the relevant mathematical scaffold required to understand dimensionality reduction.
Additionally, we became aware of the potential pitfalls that hold true to the subject in general.
Therefore, we are now ready to take a closer look on how to categorise and associate various techniques.

\subsubsection{Linear vs. non-linear problems}
\input{source/3010-linear_vs_nonlinear.tex}
\clearpage

\subsubsection{Projection vs. manifold learning}
% taken from ch 8 in géron's book

In this section, we will compare the general solution approaches available which can be utilised to solve both linear as well as non-linear problems. \cite{HandsOnMLCh8}

\todo{Not quite true, revisit this. \cite{Lee2007NonlinearDR} cite this.}

\renewcommand{\tikzscale}{0.33}
\begin{wrapfigure}[13]{r}{0.62\textwidth}
	\vspace*{-8mm}
	\centering
	\input{source/3021-projection_example.tex}
	\captionsetup{justification=centering}
	\caption{Simple example of a projection}
    \label{fig:projectionExample}
\end{wrapfigure}

\paragraph{Projection} In contrast, this is the trivial concept of the two. The idea is to project the data points onto a \gls{hyperplane} which summarises the data with as little information loss as possible.

Figure \ref{fig:projectionExample} illustrates this in a simple example.
As we can observe, when we pick the right \gls{hyperplane}, such as the x axis in the example, we lose far fewer information than if we would have picked the y axis.


\paragraph{Manifolds} This concept is significantly more difficult to get a hold of.
Significant breakthroughs \cite{ma2012manifold} in this field were accomplished in the year 2000 in the significant and commonly cited paper \emph{A global geometric framework for nonlinear dimensionality reduction}. \cite{tenenbaum2000global}
To understand the basic idea, we will demonstrate its behaviour to get an idea of the problem using the popular swiss roll data set pictured in figure \ref{fig:swissrollfull}.


\noindent
\begin{minipage}[c]{0.4\linewidth}
%
\vspace*{6mm}
\begin{center}
	\includegraphics[width=0.9\textwidth]{external_content/graphs/swiss_roll.png}
	\captionsetup{justification=centering,type=htypei}
	\captionof{figure}{Swiss Roll generated from scikit-learn \cite{scikit-learn}}
	\label{fig:swissrollfull}
\end{center}
%
\end{minipage}\hfill%
\begin{minipage}[c]{0.55\linewidth}
Before demystifying this problem, we will then dive into various methods how to bend and twist high-dimensional data into lower-dimensional spaces.

Our goal is to avoid confusing projections such as shown in figure \ref{fig:swissrollprojection}:

\begin{center}
	\includegraphics[width=0.8\textwidth]{external_content/graphs/swiss_roll-projection.png}
	\captionsetup{justification=centering,type=htypei}
	\captionof{figure}{Representation in 2D\\ of a projected swiss role}
	\label{fig:swissrollprojection}
\end{center}
\end{minipage}%

\clearpage


%%%%%%%%%%%%%%%%%%%%%%%%%%%%%%%%%%%%%%%%%%%%%%%%%%%%%%%%%%%%%%%%


\subsection{Linear techniques}
We will take a look at \gls{pca} \& \gls{lda}.


\subsubsection{Principal Component Analysis}
\begin{itemize}
	\item First approaches made in 1901 using simple projections. \cite{pearson1901liii}
	\item Primarily used for feature extraction \cite{PythonMachineLearningCh5}
	\item Unsupervised method \cite{PythonMachineLearningCh5}
	\item Is an eigenvector problem \cite{MultilinearSubspaceLearningCh2}
	\item scikit-learn implemented this version \cite{minka2000automatic} to guess the output dimesionality. \cite{halko2009finding} is used to decompose the input matrix.
\end{itemize}


\clearpage

\paragraph{full \gls{svd}} \label{svd}

According to \cite{wright2001large}

$$\bigo{N^3}$$

\clearpage

\paragraph{\gls{arpack}}

According to \cite{wright2001large}

$$\bigo{N^2}$$

\clearpage

\paragraph{randomised}

According to this \cite{HandsOnMLCh8}

$$\bigo{d^3}$$


\clearpage

\paragraph{Conclusion}

\clearpage

\subsubsection{Linear Discriminant Analysis}
\begin{itemize}
	\item First approaches made in 1936. \cite{fisher1936use}
	\item Various solvers
	\begin{itemize}
		\item \textbf{Eigenvalue Decomposition}
		\item Runs in $\bigo{N^3}$ according to \cite{cai2008training}
	\end{itemize}
	\begin{itemize}
		\item \textbf{LSQR}
		\item Runs in $\bigo{N^2}$ according to \cite{di2013new}
	\end{itemize}
		\begin{itemize}
		\item \textbf{SVD}
		\item Analogue to SVD in section \ref{svd}
	\end{itemize}
\end{itemize}

$$\bigo{TBA}$$

\clearpage


\begin{center}
	\textit{Not sure what comes here, but most certainly going to be two pages}
\end{center}

\clearpage



%%%%%%%%%%%%%%%%%%%%%%%%%%%%%%%%%%%%%%%%%%%%%%%%%%%%%%%%%%%%%%%%


\subsection{Non-linear techniques}
Mention other non-linear techniques that exist.



\subsubsection{Kernel Principal Component Analysis}
\input{source/3210-kpca.tex}
\clearpage


\subsubsection{Locally Linear Embedding}
\begin{itemize}
	\item First introduced by \cite{roweis2000nonlinear}
	\item According to this \cite{DRUnsupervisedNearestNeighbors}
	\item A manifold learning technique \cite{HandsOnMLCh8}
\end{itemize}

$$\bigo{N^2}$$

\clearpage

\begin{center}
	\textit{Not sure what comes here, but most certainly going to be two pages}
\end{center}

\clearpage


\subsubsection{Isomap Embedding}
\begin{itemize}
	\item \Gls{lle} first introduced by \cite{tenenbaum2000global}
	\item Tries to preserve the geodesic distances between the instances \cite{HandsOnMLCh8}
	\item According to this \cite{DRUnsupervisedNearestNeighbors}
\end{itemize}

$$\bigo{N^2 \log N}$$

\clearpage

\begin{center}
	\textit{Not sure what comes here, but most certainly going to be two pages}
\end{center}

\clearpage


\subsubsection{t-SNE}
\begin{itemize}
	\item \gls{tsne} first introduced by \cite{van2008visualizing}
	\item Primarily used to visualise datasets
	\item Tries to preserve the proximity between instances \cite{HandsOnMLCh8}
	\item According to this \cite{van2014accelerating}
\end{itemize}

$$\bigo{N \log N}$$

\clearpage

\begin{center}
	\textit{Not sure what comes here, but most certainly going to be two pages}
\end{center}

\clearpage
