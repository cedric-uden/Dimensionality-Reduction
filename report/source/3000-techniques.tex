Being acquainted with the premise, we now possess a coarse understanding of the relevant mathematical scaffold required to understand dimensionality reduction.
Additionally, we became aware of the potential pitfalls that hold true to the subject in general.
Therefore, we are now ready to take a closer look on how to categorise and associate various techniques.


%%%%%%%%%%%%%%%%%%%%%%%%%%%%%%%%%%%%%%%%%%%%%%%%%%%%%%%%%%%%%%%%


\subsection{Principal Component Analysis (PCA)}

\input{source/3100-pca.tex}
\clearpage


%%%%%%%%%%%%%%%%%%%%%%%%%%%%%%%%%%%%%%%%%%%%%%%%%%%%%%%%%%%%%%%%


\subsection{Non-Linear Example}
\input{source/3200-kpca.tex}
\clearpage

