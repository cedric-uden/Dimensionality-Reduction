Being acquainted with the premise, we now possess a coarse understanding of the relevant mathematical scaffold required to understand dimensionality reduction.
Additionally, we became aware of the potential pitfalls that hold true to the subject in general.
Therefore, we are now ready to take a closer look on how to categorise and associate various techniques.

\subsubsection{Linear vs. non-linear problems}
\input{source/3010-linear_vs_nonlinear.tex}
\clearpage

\subsubsection{Projection vs. manifold learning}
% taken from ch 8 in géron's book

In this section, we will compare the general solution approaches available which can be utilised to solve both linear as well as non-linear problems. \cite{HandsOnMLCh8}

\todo{Not quite true, revisit this. \cite{Lee2007NonlinearDR} cite this.}

\renewcommand{\tikzscale}{0.33}
\begin{wrapfigure}[13]{r}{0.62\textwidth}
	\vspace*{-8mm}
	\centering
	\input{source/3021-projection_example.tex}
	\captionsetup{justification=centering}
	\caption{Simple example of a projection}
    \label{fig:projectionExample}
\end{wrapfigure}

\paragraph{Projection} In contrast, this is the trivial concept of the two. The idea is to project the data points onto a \gls{hyperplane} which summarises the data with as little information loss as possible.

Figure \ref{fig:projectionExample} illustrates this in a simple example.
As we can observe, when we pick the right \gls{hyperplane}, such as the x axis in the example, we lose far fewer information than if we would have picked the y axis.


\paragraph{Manifolds} This concept is significantly more difficult to get a hold of.
Significant breakthroughs \cite{ma2012manifold} in this field were accomplished in the year 2000 in the significant and commonly cited paper \emph{A global geometric framework for nonlinear dimensionality reduction}. \cite{tenenbaum2000global}
To understand the basic idea, we will demonstrate its behaviour to get an idea of the problem using the popular swiss roll data set pictured in figure \ref{fig:swissrollfull}.


\noindent
\begin{minipage}[c]{0.4\linewidth}
%
\vspace*{6mm}
\begin{center}
	\includegraphics[width=0.9\textwidth]{external_content/graphs/swiss_roll.png}
	\captionsetup{justification=centering,type=htypei}
	\captionof{figure}{Swiss Roll generated from scikit-learn \cite{scikit-learn}}
	\label{fig:swissrollfull}
\end{center}
%
\end{minipage}\hfill%
\begin{minipage}[c]{0.55\linewidth}
Before demystifying this problem, we will then dive into various methods how to bend and twist high-dimensional data into lower-dimensional spaces.

Our goal is to avoid confusing projections such as shown in figure \ref{fig:swissrollprojection}:

\begin{center}
	\includegraphics[width=0.8\textwidth]{external_content/graphs/swiss_roll-projection.png}
	\captionsetup{justification=centering,type=htypei}
	\captionof{figure}{Representation in 2D\\ of a projected swiss role}
	\label{fig:swissrollprojection}
\end{center}
\end{minipage}%

\clearpage


%%%%%%%%%%%%%%%%%%%%%%%%%%%%%%%%%%%%%%%%%%%%%%%%%%%%%%%%%%%%%%%%


\subsection{Principal Component Analysis (PCA)}

\input{source/3100-pca.tex}
\clearpage


%%%%%%%%%%%%%%%%%%%%%%%%%%%%%%%%%%%%%%%%%%%%%%%%%%%%%%%%%%%%%%%%


\subsection{Non-Linear Example}
\input{source/3200-kpca.tex}
\clearpage

