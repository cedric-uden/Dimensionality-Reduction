Being acquainted with the premise, we now possess a coarse understanding of the relevant mathematical scaffold required to understand dimensionality reduction.
Additionally, we became aware of the potential pitfalls that hold true to the subject in general.
Therefore, we are now ready to take a closer look on how to categorise and associate various techniques.


%%%%%%%%%%%%%%%%%%%%%%%%%%%%%%%%%%%%%%%%%%%%%%%%%%%%%%%%%%%%%%%%


\subsection{Principal Component Analysis (PCA)}

\begin{itemize}
	\item First approaches made in 1901 using simple projections. \cite{pearson1901liii}
	\item Primarily used for feature extraction \cite{PythonMachineLearningCh5}
	\item Unsupervised method \cite{PythonMachineLearningCh5}
	\item Is an eigenvector problem \cite{MultilinearSubspaceLearningCh2}
	\item scikit-learn implemented this version \cite{minka2000automatic} to guess the output dimesionality. \cite{halko2009finding} is used to decompose the input matrix.
	\item \gls{pca} assumes that the data is centered around the origin \cite{HandsOnMLCh8}. scikit-learn takes care of that.
	\item \gls{svd} is a matrix factorization technique \cite{HandsOnMLCh8}
\end{itemize}


\clearpage

\paragraph{full \gls{svd}} \label{svd}

According to \cite{wright2001large}

$$\bigo{N^3}$$

\clearpage

\paragraph{\gls{arpack}}

According to \cite{wright2001large}

$$\bigo{N^2}$$

\clearpage

\paragraph{randomised}

According to this \cite{HandsOnMLCh8}

$$\bigo{d^3}$$


\clearpage

\paragraph{Conclusion}

\clearpage


%%%%%%%%%%%%%%%%%%%%%%%%%%%%%%%%%%%%%%%%%%%%%%%%%%%%%%%%%%%%%%%%


\subsection{Non-Linear Example}
\begin{itemize}
	\item First introduced by \cite{scholkopf1998nonlinear}
	\item According to this ...%\cite{}
\end{itemize}

$$\bigo{TBA}$$

\clearpage

\begin{center}
	\textit{Not sure what comes here, but most certainly going to be two pages}
\end{center}

\clearpage

