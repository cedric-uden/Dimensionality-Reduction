\subsubsection{Introduction}

\begin{itemize}
	\item Important as modern PCA algorithms are basically a statistical application of \gls{svd}
	\item Modern PCA is basically an applied version of SVD on a centered data matrix, decomposed into its most statistically descriptive factors
	\item SVD generalizes the concept of the fast Fourier transform: FFT on generic basis, SVD is tailored to the specific data
	\item Matrix factorization
	\item Provides a numerically stable matrix decomposition which is guaranteed to exist (unlike the eigendecomposition) \cite{brunton2019data}: robust and efficient
	\item Naturally occurring systems often exhibit dominant patterns
	\item Can be used to de-noise data sets (\cite{brunton2019data} - 1.1)
	\item Basis for other statistical techniques: Karhunen-Loève transform (KLT) or empirical orthogonal functions (EOFs), canonical correlation analysis (CCA)
\end{itemize}

\clearpage


%%%%%%%%%%%%%%%%%%%%%%%%%%%%%%%%%%%%%

\subsubsection{History}

\begin{itemize}
	\item Historical overview by \cite{stewart1993early}
	\item Foundation by
	\begin{itemize}
		\item First independent approaches in Italy \cite{beltrami1873sulle}
		\item Second independent approaches in France \cite{Jordan1874}	
	\end{itemize}
	\item Modern work on computational stability and effiency \cite{brunton2019data}
	\begin{itemize}
		\item Golub mentioned in Brutons work and Jolliffe's recap: \cite{golub1965calculating} \& \cite{golub1971singular}
	\end{itemize}
\end{itemize}

\clearpage


%%%%%%%%%%%%%%%%%%%%%%%%%%%%%%%%%%%%%

\subsubsection{PCA in Scikit: LAPACK / Full SVD}

\begin{itemize}
	\item Overview of the theory behind SVD
	\item scikit implements the svd from LAPACK
	\item LAPACK user guide \cite{anderson1999lapack}
	\item According to \cite{wright2001large}
	$$\bigo{N^3}$$
\end{itemize}

\ \clearpage
\ \clearpage


\paragraph{Minka's method to approximate intrinsic dimension}

\begin{itemize}
	\item scikit-learn implemented this version \cite{minka2000automatic} to guess the output dimesionality. 
	\item \cite{halko2011finding, brunton2019data} is used to decompose the input matrix.
\end{itemize}

\ \clearpage

%%%%%%%%%%%%%%%%%%%%%%%%%%%%%%%%%%%%%

\subsubsection{ARPACK / truncated SVD}

\begin{itemize}
	\item \cite{wright2001large} ARPACK approach from scikit
	\item NOT for sparse data, not reviewed here (\href{https://scikit-learn.org/stable/modules/generated/sklearn.decomposition.PCA.html}{scikit})
	\item Optimal rank-r approximation to X is given thanks to the Eckart-Young theorem \cite{eckart1936approximation}
	\item According to \cite{wright2001large, brunton2019data}
	$$\bigo{N^2}$$

\end{itemize}

\ \clearpage
\ \clearpage

%%%%%%%%%%%%%%%%%%%%%%%%%%%%%%%%%%%%%

\subsubsection{Randomized SVD}

\begin{itemize}
	\item \cite{halko2011finding} scikit implementation
	\item \cite{martinsson2011randomized} scikit implementation
	\item \cite{brunton2019data} either capture a pre determined amount of the variance or identify transition points of singular values that give represent important patterns to noise
	\item Increasingly important due to ever growing data set \cite{brunton2019data} (1.1)
	\item According to this \cite{HandsOnMLCh8}
	$$\bigo{d^3}$$
	\item Also mentioned by Brunton: \cite{sarlos2006improved}
\end{itemize}

\ \clearpage
\ \clearpage

