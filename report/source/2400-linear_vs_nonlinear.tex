Having a rough idea of the repartition of the data of a problem is important to understand to choose the right category of \gls{dr} methods.
Thus, it requires special attention by its users since it can easily be used erroneously.
We need to understand the gross patterns in the data set we need to operate on.
\bigskip

Figure \ref{fig:linearvsnonlinearproblems} below illustrates and contrasts the general patterns that we need to identify:%
\vspace*{2mm}

\renewcommand{\tikzscale}{1.2}
\begin{figure}[h]
	\begin{subfigure}{0.49\textwidth}
	    \caption{Linear problems}
		\input{source/2401-linear_problem.tex}
	    \label{subfig:linearproblems}
	\end{subfigure}
	\hfill
	\begin{subfigure}{0.49\textwidth}
	    \caption{Non-linear problems}
		\input{source/2402-non_linear_problem.tex}
	    \label{subfig:nonlinearproblems}
	\end{subfigure}
\caption{Linear vs. non-linear problems.}
\label{fig:linearvsnonlinearproblems}
\end{figure}
\bigskip

As we can see in figure \ref{subfig:linearproblems}, a linear problem is given when we are able to identify a straight line (or any \gls{hyperplane}), to split the data into unambiguous subsets.
Intuitively, these types of problem are comparably easy to solve.
\medskip

A non-linear problem, as pictured in figure \ref{subfig:nonlinearproblems}, already looks more complex on the first glance.
These methods are significantly more expensive in terms of computational resources required.
We will briefly give a high-level introduction to non-linear \acrlong{dr} at the end, as this work will primarily focus on linear problems.
