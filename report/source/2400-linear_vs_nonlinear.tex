The high-level classification of the problem is important to understand in order to chose the right category of \gls{dr} methods.

Thus, it requires special attention by its users since it can easily be used erroneously.
We need to understand the gross patterns in the data set we need to operate on.
\medskip

Figure \ref{fig:linearvsnonlinearproblems} below illustrates and contrasts the general patterns that we need to identify:\vspace*{4mm}

\renewcommand{\tikzscale}{1.18}
\begin{figure}[h]
	\begin{subfigure}{0.48\textwidth}
	    \caption{Linear problems}
		\input{source/2401-linear_problem.tex}
	    \label{subfig:linearproblems}
	\end{subfigure}
	\hfill
	\begin{subfigure}{0.48\textwidth}
	    \caption{Non-linear problems}
		\input{source/2402-non_linear_problem.tex}
	    \label{subfig:nonlinearproblems}
	\end{subfigure}
\caption{Linear vs. non-linear problems.}
\label{fig:linearvsnonlinearproblems}
\end{figure}

As we can see in figure \ref{subfig:linearproblems}, a linear problem is given when we are able to identify a straight line (or any \gls{hyperplane}), to split the data into unambiguous subsets.
Intuitively, these types of problem are comparably easy to solve.

A non-linear problem, as pictured in figure \ref{subfig:nonlinearproblems}, already looks more complex on the first glance.
