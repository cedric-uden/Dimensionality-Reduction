The utmost category, used to differentiate between various techniques is fortunately mutually exclusive to any method.
Thanks to this characteristic, we can distinctively classify, and associate a given technique.
This allocation, contrary to the following classifications, is the one that considers the domain of the problem.

Thus, it requires special attention by its users since it can easily be used erroneously.
We need to understand the gross patterns in the data set we need to operate on.
\medskip

The below figure \ref{fig:linearvsnonlinearproblems} illustrates and contrasts the general patterns that we need to identify:\vspace*{4mm}

\renewcommand{\tikzscale}{1.18}
\begin{figure}[h]
	\begin{subfigure}{0.48\textwidth}
	    \caption{Linear problems}
		\begin{tikzpicture}[scale=\tikzscale]
	% AXIS
	\draw[->,ultra thick,color=pptAccentII]
		(0,0)--(5,0) 
			node[midway, below,yshift=-1mm]{$x_1$};

	\draw[->,ultra thick,color=pptAccentII] 
		(0,0)--(0,5) 
			node[midway, left, xshift=-1mm]{$x_2$};

	% DIAGONAL LINE
	% \draw[-, dashed,color=pptAccentII] 
		% (0.2,0.2)--(4.5,4.5)
				% ;

	% NODES

	\begin{scope}[color=pptAccentIII,line width=1.25pt]
	 	\newcommand{\lowerarray}{%
	 		{2.3,1.7},{3.3,0.7},{1.2,0.5},{2.7,1.2},{2.4,1.4},{1.8,1.2},{3.4,2.4},{3.2,0.5},{1.3,0.7},{1.2,0.7},{3.7,1.8},{3.5,0.6},{2.4,0.6},{3.3,2},{1.5,0.9},{1.3,0.8},{1.3,0.7},{3.6,0.6},{3.2,2},{3.8,1.7},{1,0.5},{2.8,0.6},{3.9,2.5},{2.6,0.9},{3.7,1.7},{2.3,0.6},{3.9,1.4},{3.8,2.5},{1.4,0.7},{1.7,0.9},{3,2.4},{1.9,0.8},{2,1.5},{2.4,1.6},{1.4,0.9},{1.9,0.9},{2.8,1.7},{1.6,0.5},{3.9,0.8},{3.9,1.8},{2.3,0.7},{2.5,1.6},{1.4,0.7},{4,3.5},{3.1,0.9},{3.4,2.5},{3.3,2.2},{4,1.5},{2,1.5}%
	 	}

		\foreach \i in \lowerarray {
		 	\draw (\i) 
		 		circle (2pt);
		}
 	\end{scope}

	\begin{scope}[color=pptAccentIV,line width=1pt]
	 	\newcommand{\upperarray}{%
	 		{2.9,4},{1.7,2.7},{3.2,3.8},{2.6,3.8},{2.5,3},{3.5,4.5},{2.3,4.2},{2.8,4.3},{2.5,4.2},{4,4.5},{3.6,4.2},{3.8,4.5},{4,4.5},{1.4,4.1},{3.6,4.1},{1.2,3.7},{3.7,4.2},{3.5,4.4},{1.7,4.1},{2.3,4},{2.2,2.7},{1.4,4.2},{2.3,3.7},{4,4.5},{2.6,3.8},{1.9,2.4},{3.3,4.3},{2.8,4.4},{1.8,2.8},{1,3.1},{2.1,4.4},{1.2,3.7},{4,4.5},{2.2,3.5},{3.5,4.3},{1,1.5},{1.2,3.1},{2.2,2.9},{1.8,2.3},{1.4,4.1},{3.7,4.3},{3,3.9},{3.4,4},{1.8,3.7},{1.8,4.3},{2.5,4},{3.7,4.4},{2.8,3.8},{1.5,4.1}%
	 	}

		\foreach \i in \upperarray {
		 	\node[isosceles triangle,draw,isosceles triangle apex angle=60,rotate=90, minimum size=3pt,inner sep=0pt] (T) at (\i){};
		}

	 \end{scope}

\end{tikzpicture}

	    \label{subfig:linearproblems}
	\end{subfigure}
	\hfill
	\begin{subfigure}{0.48\textwidth}
	    \caption{Non-linear problems}
		\begin{tikzpicture}[scale=\tikzscale]
	% AXIS
	\draw[->,ultra thick,color=pptAccentII] 
		(0,0)--(5,0) 
			node[midway, below, yshift=-1mm]{$x_1$};

	\draw[->,ultra thick,color=pptAccentII] 
		(0,0)--(0,5) 
			node[midway, left, xshift=-1mm]{$x_2$};

	% DELIMITER CIRCLE
 	% \draw (2.5,2.5)[dashed,color=pptAccentII]
 		% circle (42pt);

	% NODES

	\begin{scope}[color=pptAccentIII,line width=1.25pt]
	 	\newcommand{\innerarray}{%
	 		{2.4,2.5},{1.8,1.8},{3,3.4},{3.4,2.1},{1.9,2.8},{1.6,2.1},{2,1.6},{2.7,2.7},{3,3.5},{3,2.4},{2.6,1.5},{2,2.6},{2.7,3},{1.5,3},{2.6,2.7},{3.1,2},{1.9,3.4},{2,2.6},{3,2.2},{1.6,3.2},{2.8,1.5},{2.3,2.7},{2.6,2.5},{2.5,3.5},{3.3,1.7},{2.1,1.8},{2.2,2.1},{1.6,2.8},{3.1,2.9},{1.7,2.2},{1.6,2.9},{2.7,2.7},{2,1.8},{3,1.7},{3.4,2.7},{3.4,2.3},{2.1,2.6},{1.7,1.7},{2.9,3.2},{1.6,3},{3,2.9},{2.4,2.6},{2.5,2.8},{3.3,3.2},{2.9,2.2},{2.3,2.2},{2.8,2.7},{2,2.5},{3.1,1.8}%
	 	}

		\foreach \i in \innerarray {
		 	\draw (\i) 
		 		circle (2pt);
		}
 	\end{scope}

	\begin{scope}[color=pptAccentIV,line width=1pt]
	 	\newcommand{\outerarray}{%
	 		{0.2,0.6},{0.9,0.9},{1.3,0.4},{0.4,0.5},{0.7,1.1},{1.2,0.2},{1.3,0.4},{0.3,0.4},{0.5,1.4},{0.9,1.2},{0.5,0.4},{0.3,0.3},{4.3,4.3},{4.5,4.1},{4.2,4.5},{3.9,4.4},{3.7,4.5},{3.9,3.7},{4.4,4.1},{4.2,4.4},{3.9,3.7},{3.9,4},{3.8,3.6},{4,4},{4.4,0.6},{4.1,1.4},{3.7,1.1},{3.9,0.6},{4.3,0.3},{4.2,1.4},{4.2,0.7},{3.8,1.3},{4.1,0.7},{4.1,1.1},{4.3,1.2},{4.3,1.4},{0.2,4.5},{1.1,4.3},{1.2,4.5},{0.8,3.6},{0.8,3.6},{0.4,4.2},{0.5,4.2},{0.8,3.7},{0.2,3.5},{0.4,4.3},{0.9,3.8},{0.5,4.5},{3.2,0.8},{3.5,0.8},{3.3,0.8},{2.7,0.8},{2.1,0.5},{2.3,0.6},{2,0.6},{3.4,0.8},{3.1,0.5},{2.7,0.8},{3.5,0.7},{2.6,0.5},{0.8,3.1},{0.7,2.4},{0.7,3.1},{0.8,2.6},{0.6,2.1},{0.5,2.9},{0.8,2.2},{0.7,2.9},{0.8,2.4},{0.7,3.1},{0.8,3.4},{0.7,2.2},{3.4,4.4},{2.8,4.2},{2.4,4.5},{3.5,4.5},{3.4,4.2},{3.1,4.5},{2.9,4.3},{2.5,4.5},{3.4,4.5},{2.2,4.2},{3.1,4.5},{2.7,4.5},{4.5,3},{4.5,2.7},{4.3,2.3},{4.2,2.3},{4.2,2.9},{4.5,2.5},{4.2,2.6},{4.4,2.1},{4.2,2.1},{4.2,2},{4.3,3.4},{4.4,2.2}%
	 	}

		\foreach \i in \outerarray {
		 	\node[isosceles triangle,draw,isosceles triangle apex angle=60,rotate=90, minimum size=3pt,inner sep=0pt] (T) at (\i){};
		}

	 \end{scope}

\end{tikzpicture}

	    \label{subfig:nonlinearproblems}
	\end{subfigure}
\caption{Linear vs. non-linear problems.}
\label{fig:linearvsnonlinearproblems}
\end{figure}

As we can see in figure \ref{subfig:linearproblems}, a linear problem is given when we are able to identify a straight line, or any \gls{hyperplane}, to split the data into unambiguous subsets.
Intuitively and, as we will later examine, factually, these types of problem are comparably easy to solve.

A non-linear problem, as pictured in figure \ref{subfig:nonlinearproblems}, already looks more complex on the first glance.
And indeed, we will illuminate the solution approaches to this scenario and elaborate on its comportment.
