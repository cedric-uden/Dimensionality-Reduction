We are going to get an idea on how various applications of dimensionality reduction look like and how they behave.

% \subsection{Improve computational performance}

% This one is the most obvious.
% We are going to find examples of projects / use cases where the performance is greatly improved without affecting the models performance.

% Maybe the MNIST dataset?

% \clearpage



\subsection{Eigenfaces}

\begin{itemize}
  \item Talk about grouping similar areas
  \item Compression example
\end{itemize}

\clearpage





\subsection{Visualisation}

Showcase how data can be represented in different ways.

Maybe see 33c3 (\href{https://www.youtube.com/watch?v=-YpwsdRKt8Q}{Reverse Engineering von Spiegel-Online (33c3)})

Should look something like this:

\begin{figure}[h]
  \centering
  \includegraphics[width=0.9\linewidth]{external_content/graphs/latent_space_reduction_example.png}
  \captionsetup{justification=centering}
  \captionof{figure}{Visual effects from a dimensionality reduction}
  \label{fig:visualisingReduction}
\end{figure}
\todo{This is not my work! This is a placeholder and will be replaced in the final version}

% talk about how the swiss roll would look like with PCA

\clearpage


