% \gls{ } : This command prints the term in lowercase characters. For example, \gls{glsy} will print glossary.
% \Gls{ } : This is similar to \gls{ } command. The only difference is it will print the first character in the uppercase For example, \Gls{glsy} will print Glossary.
% \glspl{ } : This command is similar to \gls{ }. The only difference is that it will convert the term into its plural form.
% \Glspl{ } : This command is similar to \Gls{ } command, with the difference that it will convert the term in its plural also.

% \newglossaryentry{}
% {
%     name=,
%     description={}
% }

\newglossaryentry{scikit}
{
    name=scikit-learn,
    description={Scikit-learn is an open source machine learning library that supports supervised and unsupervised learning... \href{https://scikit-learn.org/stable/getting_started.html}{Link to the full documentation}}
}

\newglossaryentry{hyperplane}
{
    name=hyperplane,
    description={A hyperplane is a subspace whose dimension is one less of the space it is currently represented in. e.g. a one-dimensional line in a two-dimensional graph}
}

\newglossaryentry{latentVariable}
{
    name=latent variable,
    description={A latent variable is one that is not directly observed or measured \cite{rosipal2005overview}}
}

\newglossaryentry{intrinsicDimension}
{
    name=intrinsic dimension,
    description={The low-dimensional manifold in which the high-dimensional data can be reduced without losing much information \cite{GeometricStructureWangCh1}}
}

\newglossaryentry{qr}
{
    name=$QR$ decomposition,
    description={$QR$ factorisation method to produce a low-rank approximation. Allows to express an input matrix $A$ as the product of an orthogonal matrix $Q$ and a right-triangular factor $R$ such that $A = QR$ \cite{duersch2017randomized}}
}

\newglossaryentry{onb}
{
    name=orthonormal basis,
    description={all basis vectors are orthogonal to each other and the length of each basis vector is 1 \cite{deisenroth2020mathematics}}
}

\newglossaryentry{lineintegral}
{
    name=line integral,
    description={Instead of integrating over an interval $[a,b]$, we integrate over a curve $C$.
    \emph{Curve integral} would be a more intuitive terminology. Generally, the line integral could be rewritten as following:\medskip
    \\
    $\int_C f(x,y) ds = \int_a^b f(x(t), y(t)) \sqrt{(\frac{dx}{dt})^2 + (\frac{dy}{dt})^2} dt$
    \quad \cite{stewart2016calculus}}
}
