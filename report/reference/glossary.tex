% \gls{ } : This command prints the term in lowercase characters. For example, \gls{glsy} will print glossary.
% \Gls{ } : This is similar to \gls{ } command. The only difference is it will print the first character in the uppercase For example, \Gls{glsy} will print Glossary.
% \glspl{ } : This command is similar to \gls{ }. The only difference is that it will convert the term into its plural form.
% \Glspl{ } : This command is similar to \Gls{ } command, with the difference that it will convert the term in its plural also.

\newglossaryentry{scikit}
{
    name=scikit-learn,
    description={Scikit-learn is an open source machine learning library that supports supervised and unsupervised learning... \href{https://scikit-learn.org/stable/getting_started.html}{Link to the full documentation}}
}

\newglossaryentry{hyperplane}
{
    name=hyperplane,
    description={A hyperplane is a subspace whose dimension is one less of the space it is currently represented in. e.g. a one-dimensional line in a two-dimensional graph}
}
